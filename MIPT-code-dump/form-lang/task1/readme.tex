\documentclass[a4paper,12pt]{article}
\usepackage{standalone}

\usepackage[utf8]{inputenc}
\usepackage[russian]{babel}
\usepackage{mathtools}
\usepackage{amssymb}
\usepackage{amsmath}
\usepackage{amsthm}
\begin{document}

\section*{Задача}

Даны буква $x,$ натуральное $k$ и регулярное выражение $\alpha$ в обратной польской записи.
Вывести минимальную длину слова, входящего в $L(\alpha)$ и начинающегося с $x^k,$ или \texttt{INF},
если такого слова не существует.

\section*{Описание алгоритма}

Преобразуем регулярное выражение в недетерменированный конечный автомат со следующими свойствами:
\begin{enumerate}
    \item у него ровно одна конечная вершина;

    \item у него все переходы --- либо $\varepsilon-$переходы, либо однобуквенные.
\end{enumerate}

После этого посчитаем ответ методом динамического программирования по парам (вершина автомата, значение $k$ из
условия).

Подробнее, пусть $f(v,k)$ --- минимальная длина слова, достижимого из вершины $v$ и содержащего $x^k$ в качестве
префикса; $S$ --- стартовая вершина нашего автомата; $T$ --- конечная вершина нашего автомата.
Тогда
$$
f(v,k)=
\begin{cases}
 \min\limits_{v \xrightarrow{\alpha \in \{\varepsilon, x\}} w} f(w, k - |\alpha|) + |\alpha|, & k \ne 0; \\
 \min\limits_{v \xrightarrow{\alpha} w} f(w, 0) + |\alpha|,                                   & k = 0 \land v \ne T; \\
 0,                                                                                           & k = 0 \land v = T.
\end{cases}
$$

Подразумевается, что $\min \varnothing = +\infty.$

Стоит отметить, что при вычислении $f$ проходить по циклу из вершин без изменения второго параметра $f$ не имеет
никакого смысла; поэтому, при вычислении значения $f$ для очередной пары $(v,k)$ нужно запоминать,
что значение для этой пары находится в процессе вычисления (и стирать этот флаг при выходе из функции);
а если при вызове $f$ получилось так, что вычисление значения для этой пары $(v,k)$ уже находится в процессе вычисления
(то есть, получился цикл), возвращать $+\infty.$

Ответом будет являться $f(S,k),$ где $k$ --- $k$ из условия.

\end{document}
